\documentclass{beamer}

\usepackage{hyperref}
\usepackage[latin1]{inputenc}
\usepackage{graphicx}
\usepackage{listings}
\usepackage{color}

\usetheme{Copenhagen}
\usecolortheme{beetle}

% PDF Properties
\hypersetup
{
    pdfauthor={Jorn Jambers, Wout Decr\'{e} \& Jan Collijs},
    pdfsubject={Kickstart},
    title={Projectwerk - Presentatie},
}

\begin{document}

\lstset{ %
basicstyle=\tiny,       % the size of the fonts that are used for the code
numbersep=5pt,                  % how far the line-numbers are from the code
breaklines=true,                % sets automatic line breaking
breakatwhitespace=false,        % sets if automatic breaks should only happen at whitespace
backgroundcolor=\color{white}
}



	\title{Kickstart \thanks{Presentatie in kader van het projectwerk.}}
	\author{Jorn Jambers, Wout Decr\'{e} \& Jan Collijs}
	\institute{Katholieke Hogeschool Leuven}
	\date{\today} 

	\frame{\titlepage} 

	\frame{\tableofcontents} 

	\section{Inleiding}
	\subsection{Beschrijving}
	\frame{
		
		\frametitle{Beschrijving}
		\begin{itemize}
		\item <2-> Red hat
		\item <3-> Unattended installie windows
		\end{itemize}
		
	}

	\subsection{Automatisatie}
	\frame{
		
		\frametitle{Automatisatie}
		\begin{itemize}
		\item <2-> Taal
		\item <3-> Partionering
		\item <4-> Netwerk configuratie
		\item <5-> Extra packages
		\item <6-> ...
		\end{itemize}
	}
	\subsection{Voordelen}
	\frame{
		
		\frametitle{Voordelen}
		\begin{itemize}
		\item <2-> Snelle interventie
		\item <3-> Hardware onafhankelijk
		\item <4-> Meerdere malen dezelfde installatie
		\end{itemize}
		
	}
	\subsection{Waarom kickstart}
	\frame{
		
		\frametitle{Waarom kickstart}
		\begin{itemize}
		\item <2-> "Spread the word"
		\item <3-> Flexibele opstelling
		\end{itemize}
		
	}
	
	\section{Configuratie}	
	\subsection{isolinux.cfg}
	\begin{frame}[fragile]
		isolinux/isolinux.cfg
		\begin{lstlisting}
			DEFAULT /install/vmlinuz
			GFXBOOT bootlogo
			APPEND  file=/cdrom/preseed/ubuntu-server.seed initrd=/install/initrd.gz quiet --
			LABEL installslave
			  menu label ^Install Ubuntu Server as Puppet slave
			  kernel /install/vmlinuz
			  append  ks=cdrom:/ksslave.cfg fb=false file=/cdrom/preseed/ubuntu-server.seed initrd=/install/initrd.gz quiet --
			LABEL installmaster
			  menu label ^Install Ubuntu Server as Puppet master
			  kernel /install/vmlinuz
			  append  ks=cdrom:/ksmaster.cfg fb=false file=/cdrom/preseed/ubuntu-server.seed initrd=/install/initrd.gz quiet --
			LABEL linux
			  menu hide
			  kernel /install/vmlinuz
			  append  file=/cdrom/preseed/ubuntu-server.seed initrd=/install/initrd.gz quiet --
			LABEL cdrom
			  menu hide
			  kernel /install/vmlinuz
			  append  file=/cdrom/preseed/ubuntu-server.seed initrd=/install/initrd.gz quiet --
		\end{lstlisting}
	\end{frame}
	
	\begin{frame}[fragile]
		\begin{lstlisting}
			LABEL expert
			  menu hide
			  kernel /install/vmlinuz
			  append  file=/cdrom/preseed/ubuntu-server.seed priority=low initrd=/install/initrd.gz --
			LABEL check
			  menu label ^Check CD for defects
			  kernel /install/vmlinuz
			  append  MENU=/bin/cdrom-checker-menu initrd=/install/initrd.gz quiet --
			LABEL rescue
			  menu label ^Rescue a broken system
			  kernel /install/vmlinuz
			  append  rescue/enable=true initrd=/install/initrd.gz --
			LABEL memtest
			  menu label Test ^memory
			  kernel /install/mt86plus
			  append -
			LABEL hd
			  menu label ^Boot from first hard disk
			  localboot 0x80
			  append -
			DISPLAY isolinux.txt
			TIMEOUT 0
			PROMPT 1
			F1 f1.txt
			F2 f2.txt
			...
	\end{lstlisting}
	\end{frame}

	\subsection{ks.cfg}
	\begin{frame}[fragile]
		root directory
		\begin{lstlisting}
			#Generated by Kickstart Configurator
			#platform=x86
			#System language
			lang en_US
			#Language modules to install
			langsupport en_US
			#System keyboard
			keyboard be
			#System mouse
			mouse
			#System timezone
			timezone Europe/Brussels
			#Root password (= $$32Puppet)
			rootpw --iscrypted $1$HUUqY2cv$EDqnPg0QxtPBIM9d22ynC/
			#Initial user
			user --disabled
			#Reboot after installation
			reboot
			#Use text mode install
			text
			#Install OS instead of upgrade
			install
			#Use CDROM installation media
			cdrom
			#System bootloader configuration
			bootloader --location=mbr 
			#Clear the Master Boot Record
			zerombr yes
			
		\end{lstlisting}
	\end{frame}
	
	\begin{frame}[fragile]
		\begin{lstlisting}
			#Partition clearing information 
			clearpart --all --initlabel
			#Disk partitioning information 
			part / --fstype ext3 --size 1 --grow 
			part swap --size 1024 
			#System authorization infomation
			auth  --useshadow  --enablemd5 
			#Network information
			network --bootproto=dhcp --device=eth0
			#Firewall configuration
			firewall --disabled 
			#Do not configure the X Window System
			skipx
			%packages
			openssh-server
			dialog
			mysql-server
			python-mysqldb
			puppet
			puppetmaster
			%post --interpreter=/bin/bash
			#Capture input from tty6, and send output to tty6
			chvt 6
			exec < /dev/tty6 > /dev/tty6
			 
			clear
			 
			/media/cdrom/puppet/master.sh
			#Capture input from tty1, and send output to tty1
			chvt 1
			exec < /dev/tty1 > /dev/tty1
		\end{lstlisting}
	\end{frame}
	
	\subsection{Splash screen}
	\frame{
		
		\frametitle{Splash screen}
		isolinux/splash.pcx\\
		\includegraphics[scale=0.3]{./img/PuppetSplash.PNG}	
	}
	
	\subsection{Kickstart configurator}	
	\frame{
		
		\frametitle{Kickstart configurator}
		\includegraphics[scale=0.3]{./img/Kickstartconfigurator.PNG}
	}

	\section{Demonstratie}
	\frame{
		\frametitle{Demo}
		Kickstart iso image aanmaken mbv iso-master.\\
		Installatie van een besturingssysteem mbv kickstart.
	}
	
	\section{Bronnen}
	\frame{
		\frametitle{Bronnen}
		\begin{itemize}
			\item \href{http://en.wikipedia.org/wiki/Kickstart\_(Linux)}{Wikipedia - Kickstart}
			\item \href{http://en.wikibooks.org/wiki/LaTeX/Presentations}{Wikibooks - Latex ;)}
		\end{itemize} 
	}
	
	\section{Slot}
	\frame{
		\frametitle{Slot}
		\begin{itemize}
			\item <1-> Vragen
			\item <2-> Bedankt
		\end{itemize}
	}
\end{document}
